\chapter{General Approach to the Problem}
\label{cha:approachestotheproblem}

In this chapter, our approach to find and repair conflicting spots in the netlists is described.

\section{Routing Elements in RapidSmith}
\label{sec:routingelementsinrapidsmith}

Since the RapidSmith API define multiple elements involved in the routing process, a short overview over these elements is given in the following list:
\begin{itemize}
\item \textbf{Wire}\hfill \\
A \texttt{wire}is the basic connection element. It is represented as an integer due to the large number of wires on an FPGA. Wires can be hard connected to other wires, which can be determined through the \texttt{Device} class.
\item \textbf{Pin}\hfill \\
A \texttt{Pin} connects an input or output of a logic block to routing resources. It is primarily connected to one \texttt{wire}, and may be defined as output or input pin.
\item \textbf{PIP}\hfill \\
A PIP object describes an active connection between two wires, which are attributes to the object. If it is contained in a \texttt{net}, is is switched on. Otherwise, it remains turned off.
\item \textbf{Net}\hfill \\
A Net gathers a list of connected pins and active PIPs. It also has a defined source pin driving the net.
\item \textbf{WireConnection}\hfill \\
A WireConnection is a wrapper class for a \texttt{wire} and holds the information whether this wire is only reacheable if a pip is turned on. This class is used to get information about the connectable wires on the FPGA.
\item \textbf{Node}\hfill \\
A Node is a routing object used in the router package. It can be used to describe routes of \texttt{wires} and \texttt{PIPs} in a router.
\item \textbf{SinkPin}\hfill \\
A SinkPin refers to a switch matrix on the FPGA. Nearly all pins' wires are connected to a switch matrix, which is an interface to the other wires
\item \textbf{Tile}\hfill \\
A Tile is one of the checkerboard-lie distributed areas on an FPGA. It holds various elements from the list above and a set of logic elements. 

\end{itemize}

\section{Isoelectric Potential Search}
\label{sec:isolectricelements}

Each design's routing in RapidSmith can be viewed as a set of \texttt{Net} objects. According to the RapidSmith documentation \cite{techDoc}, these \texttt{Net} objects hold all used pins and pips of the net.
Therefore it should be possible to reach all sink pins of a net by following wires connected to the input pin if the net is not broken.
Consequently each design is broken if it is not possible to reach the output pins of an net from the input pin.  

A \texttt{Net} in RapidSmith only contains \texttt{PIP} objects representing PIPs that are currently "switched on". Therefore, a net can be expanded if additional PIPs are switched on. 
It is theorized that it is possible to fix the broken net by switching on PIPs which reconnect the net's pins' connected wires with each other. This can only be done when the PIP will not connect the circuit to an independent third circuit.

To check whether or not a net can be traversed from it's source pin to all output pins an new measurement called \texttt{Potential} is introduced into RapidSmith.
A \texttt{Potential} can be seen as the isoelectric set of wires, pins and PIPs which are connected, starting from one pin.
It is never possible for a \texttt{Pin}, \texttt{PIP} or \texttt{Wire} to be in two \texttt{Potential}s at the same time. It is not possible that two non-connected elements have the same \texttt{Potential}. 
This concept is named after the concept of isoelectric potentials found in classical electronics. Although two electrical potentials may be at the same numeric voltage level and still be different (not isoelectric), this is not the case for the introduced potentials.

Before any further operation on a \texttt{Potential} may be performed, its spatial spread must be computed. This means that, starting from a pin, each electrically connected wire, pin and pip must be found. The search for connected elements works as described in algorithm \ref{alg:findindpotential}. It is also pointed graphically in Figure \ref{fig:buildpotential}.

\begin{algorithm}[h]
	wires = \{wireOf(sourcePin)\};\\
	pips = \{\};\\
	pins = \{sourcePin\};\\
	adjacentPIPs = \{\};\\
	\While{new elements added}{
	\ForEach{wire i}{
	  \ForEach{reacheableWire(i)}{
			\If{!reacheable.isPIP()}{
				add reacheableWire to wires;\\
			}{
				\ElseIf{pip \in net.pips()}{
					add reacheableWire to wires;\\
					add pip to pips;\\
				}
				\Else{
					add pip to adjacentPIPs;\\
					}
			}
	   	}
	  }
	}
	
		\ForEach{wire i}{
		add connected pin to pins;\\
		}
 \caption{Algorithm to determine all elements on one isoelectric potential.}
 \label{alg:findingpotential}
\end{algorithm}

\begin{figure}
\includegraphics[scale=0.1]{images/Handzeichnung.jpg}
\caption{Graphical visualization of the \texttt{Potential} derivation.}
\label{fig:buildpotential}
\end{figure}

Given this method, a \texttt{Potential} derived for one pin must always hold any other pin in the net if the net is routed correctly.

\texttt{Potential}s can be fused by setting PIPs. Electrically, this means that two isoelectric sets of elements are connected by a switch, and thus become one isoelectric potential. For the obejct representation this means that, if a pip is set, the two \texttt{Potential} objects have to be united.



\section{Finding and Reparing broken Nets}
\label{sec:findingandrepairingbrokennets}

In order to check if an net is broken the potential of each pin is calculated and compared. If the net contains more than one unique potential then the net is broken.
In that case it is necessary to reconnect the two parts (potentials) of that net.If the assumption that only one or two PIPs are missing is true, this can be done by the applying the following simply breadth-first-search on the \texttt{Potential}s of pins A and B:

Let \textit{leaves} be an order priority queue which holds only PIPs that do not connect to other potential besides B. \textit{leaves} is ordered by the number of wires between the PIP and Potential A.

Put into \textit{leaves} all adjacent PIPs of A.
While \textit{leaves} not empty do:
	Get first PIP of \textit{leaves}
	If PIP has Potential B then done
	Add all adjecenten PIPs to leaves

c