\chapter{Introduction}
\label{cha:introduction}

Routing and Placement are two necessary and time-consuming steps in the FPGA synthesis toolchain. These processes can sometimes be optimized or speed up to enhance the FPGA design process significantly. The RapidSmith\cite{lavin2010rapid} Java framework is a freely available tool that helps with these steps. It offers an API to read in Xilinx designs and FPGA descriptions, and offers researchers the possibility to approach placement and routing problems in an object-oriented way. In RapidSoC\cite{wenzel2016rapidsoc}, a project based on RapidSmith, a re-placement method for already placed and routed modules is integrated. Although this method works for the majority of modules, some modules routing elements cannot be re-placed due to irregular structures in the FPGA fabric. In this project, the cause for these errors is examined, and an approach to fix them is introduced.

The report is structured as following: First, the exact problem description is presented. The designed approach to the problem using isoelectric wires is described. After that, the insights gained are presented. Finally, future work is proposed, and the report is concluded.


\section{Task Description}
\label{sec:taskdescription}
To speed up the FPGA development cycle it can be preferable to move a fixed design on an FPGA from one location to another without the process of rerouting the design. If the FPGA is regular, like the first iterations of FPGAs generally were, this process can be done without any problem. Unfortunately, FPGA vendors introduced some irregularities in their designs to improve the efficiency of the FPGA and lower the costs. 
Due to this it is possible that an FPGA design that was moved on the FPGA is no longer operational after the re-placement.
The task of this project is to check if a design is broken and repair the design if possible.
It is suspected that a corrupted part of the FPGA only contains a small amount of irregularities (about 1-2 programmable interconnect points ("PIPs")) and detection and repair of the conflicting nets is faster than new routing. This work makes use of the interconnect description features of RapidSmith, which operate on a mask of the FPGA fabric.
%Historically the fabric of FPGAs was generally regular. In such a FPGA it is possible to move a .In recent years the FPGA vendors introduced some irregularities to improve the design layout and cost efficiency of the FPGA. 
%Since the fabric of FPGAs is in general regular, the expected error source for designs at different locations are irregularities regarding the routing elements in the fabric. In the current state, the nets of the design have to be re-routed, which is time-consuming. Since we guess that only 1-2 programmable interconnect points ("PIPs") are set falsely, a detection and repair of the conflicting nets promises faster success than a new routing. The task for this project is to determine a method which finds places where the described error occurs and is able to include the missing PIPs into the net. The proposed method makes use of the interconnect description features of RapidSmith, which operate on a mask of the FPGA fabric.

\section{Framework and Environment}
\label{sec:frameworkandenvironment}

This project is embedded into the RapidSmith Java environment, version a.b.c. It uses classes of the RapidSoC system. Especially the class \texttt{MoveModulesEverywhere} acts as the basis of our work. \texttt{MoveModulesEverywhere} parses existing designs from \texttt{.xdl} files into the RapidSmith environment, places the included modules on any feasible destination, and generates the corresponding \texttt{.xdl} output files. Further processing is accomplished with Xilinx ISE \cite{ise}. Based on the output of the ISE tool \texttt{xdl}, the re-placed designs can be evaluated to be functional or conflicting.

All processing developed in this project is performed in a separate Java package. Every re-placed design is extracted and examined before the output step, so that the developed methods can be applied to a RapidSmith \texttt{Design} object.

\subsection{Naming Conventions}
\label{subsec:namingconventions}

Throughout this report, multiple references to nets, pins, PIPs, tiles and nodes are made. These describe the physical objects on the FPGA. \texttt{Net, Pin, PIP, Tile} and \texttt{Node} describe their object representation in RapidSmith, which is mostly equally named as the physical object. 