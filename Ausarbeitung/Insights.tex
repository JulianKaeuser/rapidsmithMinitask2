\chapter{Insights}
\label{cha:insights}

Although the approach described in chapter \ref{cha:approachestotheproblem} seems promising, testing it for various example nets (conflicting and functional in terms of the ISE \texttt{xdl} output) did not lead to the expected results. Therefore, we examined the reasons, implementation and other possible error sources.

\section{Hand Routing}
\label{sec:handrouting}

The RapidSmith framework has some built-in example classes, which are designed to help the user understand the way RapidSmith works. One of these classes is the \texttt{HandRouter}, which allows the routing of a net through a console. It displays reacheable wires at the next iteration and adds PIPs as the route is chosen. We used and adapted the \texttt{HandRouter} to manually examine broken nets. Working on complete \texttt{.xdl}-files only, we adapted the router to work on single nets, and inserted it after the potential derivation.

Using the HandRouter reveals several error sources, but also that each connection reacheable from each pin is correctly integrated in the pin's \texttt{Potential} instance. 

\section{Possible Solutions}
\label{sec:possibleSolutions}

For the encountered problems, we propose the following approaches

