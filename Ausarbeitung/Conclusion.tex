\chapter{Conclusion}
\label{cha:conclusion}

In conclusion it seems that there is currently no possibility to determine the exact properties of an net with RapidSmith based on the '"xdl'" files. Therefore there is no way to determine which net might be broken or fix the broken design.
During our experiments with the RapidSmith hand router we started to suspect that our database might be to small. Most nets, even those of not broken designs e.g. designs that can be used without a problem by the ISE tool-chain, contain no wires or other elements connecting the pins with each other. The few Nets that are connected are usually of a rather simplistic manner.
%Alternatively there is the possibility that RapidSmith might not be able
Alternatively there is the slim possibility of bugs within RapidSmith. While we did not encounter any evidence for bugs we must note that the RapidSmith version used in this Project showed some differences to the publicly available version on http://rapidsmith.sourceforge.net/. While most of this differences seem to be improvements, for example the wireEnumerator was upgraded by one version, it should also be noted that the GUI seems to be missing files.

All things consider we recommend to use the algorithmic approach of this work with a bigger dataset or an alternative tool.


  http://rapidsmith.sourceforge.net/papers/Nelson-FPL11-Presentation.pdf

We suspect As shown in REF are X to Y big and ours are simply

the task was to
- examine re-placed designs' net lists

- find out if something does not work

- expected: only one or two PIPs are missing due to irreguar structure 
of FPGA

- if problem is encountered, find rule/mask based method to fix these 
missing pips


what we did was:

- handle net by net 

- for each Pin, determine all elements of the isoelectric it is 
connected to (class potential)

- checking method: if (potential(source)!=potential(sink) -> broken


- for broken nets:

 - search isoelectric of source, sink for adjacent, non-set PIPs 
(interesction of sets). if !=empty set, activate this pip


- why it did not work:

  - even for correct (in terms of make file) nets, the method fails

  - hand routing (function of RapidSmith framework) neither does

  - we suspect missing information in fabric;

  - according to the hand router, the method of "isoelectric search" 

results in correct sub nets; might be useful for later work
